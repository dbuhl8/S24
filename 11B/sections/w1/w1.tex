\documentclass{beamer}

\title{Section Week 1}
\author{TA: Dante Buhl}
\institute{UCSC AM-11B}
\graphicspath{ {./images/} }
\usepackage{verbatim}
\usetheme{Dresden}
\usecolortheme{spruce}
\usefonttheme{serif}
\usepackage{amsmath}
\usepackage{pifont}
\usepackage{hyperref}


\begin{document}

\newcommand{\bmp}[1]{\begin{minipage}{#1\textwidth}}
\newcommand{\emp}{\end{minipage}}

\frame{\titlepage}

\section{Agenda}
\begin{frame}{Plan for Today}
    Topics to Cover
    \begin{itemize}
        \item Introduction
        \item Differentiation Review
        \item Antiderivatives and Integrals
    \end{itemize}
\end{frame}

\begin{frame}{Learning Outcomes}
    \begin{itemize}
        \item Connecting the ideas of differentiation and integration
        \item Understanding the notation used for anti-derivatives and integrals.
    \end{itemize}
\end{frame}

\section{Activity / Review}

\begin{frame}{Derivative Practice!}

    \begin{enumerate}

        \item Power Rule\[\frac{d}{dx} x^9 + 2x^2 + 4200000x\]

        \item Chain Rule\[\frac{d}{dx} \ln(x^9 + 2x^2 + 4200000x)\]

        \item Product Rule\[\frac{d}{dx} 4x^3e^x\]
    
        \item Quotient Rule\[\frac{d}{dx} \frac{x+2}{\sqrt{x}}\]

        \item Chain Rule\[\frac{d}{dx} \sqrt[3]{4x + x^3}\]

    \end{enumerate}

\end{frame}

\begin{frame}{Antiderivatives / Integrals}

    Lets dissect the basic notation!

    \[
        \int f(x) dx, \quad \int_a^b f(x) dx
    \]
    
    The example can be read as, ``the integral of $f(x)$ with respect to $x$ from $a$ to $b$''.

    \begin{itemize}

        \item $f(x)$ is denoted the ``integrand''

        \item $dx \sim$ ``infinitesimal'', indicates which variable to integrate with

        \item $a, b$ are the lower and upper ``limits of integration''

    \end{itemize}

    Note: the left is an indefinite integral, and the right is a definite integral.

\end{frame}


\begin{frame}{Antiderivatives?????}

    There is a basic relationship between derivatives and integrals. This is summarized by the Fundamental Theorem of Calculus. We have, 
    
    \[
        \frac{d}{dx} F(x) = f(x), \quad \text{f is the derivative of F}
    \]

    \[
        \int f(x) dx = F(x) + C, \quad \text{F is the antiderivative of f}
    \]

\end{frame}

\begin{frame}{Lets Practice}

    \begin{enumerate}

        \item \[\int 9x^8 + 4x + 4200000 dx\]

        \item \[\int \frac{9x^8 + 4x + 4200000}{x^9 + 2x^2 + 4200000x}dx\]

        \item \[\int e^x\left(12x^2 + 4x^3\right)dx\]
    
        \item \[\int \frac{1}{2}x^{-\frac{1}{2}} - x^{-\frac{3}{2}}\]

        \item \[\int \frac{4 + 3x^2}{3\left(4x + x^3\right)^{2/3}}\]

    \end{enumerate}

\end{frame}

\end{document}
