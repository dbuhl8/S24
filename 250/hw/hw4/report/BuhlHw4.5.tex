\documentclass{article}
\usepackage{graphicx} % Required for inserting images
\usepackage[margin=1in]{geometry}
\usepackage{amsmath}
\usepackage{amsthm}
\usepackage{amssymb}
\usepackage{amsfonts}
\usepackage{enumitem}
\usepackage{verbatim}
\usepackage{xcolor}
\usepackage{soul}

\title{Homework 4.5: Report}
\author{Dante Buhl}

\DeclareMathOperator{\cond}{cond}
\DeclareMathOperator{\vecspan}{span}

\begin{document}

\newcommand{\bs}[1]{\boldsymbol{#1}}
\newcommand{\bmp}[1]{\begin{minipage}{#1\textwidth}}
\newcommand{\emp}{\end{minipage}}
\newcommand{\R}{\mathbb{R}}
%\newcommand{\Imag}{\mathbb{I}}
\newcommand{\C}{\mathbb{C}}
\newcommand{\N}{\mathcal{N}}
\newcommand{\I}{\mathrm{I}}
\newcommand{\K}{\bs{\mathrm{K}}}
\newcommand{\m}{\bs{\mu}_*}
\newcommand{\s}{\bs{\Sigma}_*}
\newcommand{\dt}{\Delta t}
\newcommand{\tr}[1]{\text{Tr}(#1)}
\newcommand{\Tr}[1]{\text{Tr}(#1)}

\maketitle

\section*{Question 1: PCAM for Finite Difference Stencil Problem}
\begin{enumerate}

    \item[\textbf{P} -] Smallest task is computing the derivative / ``other
    physics'' for one point in the domain. Different processors can work on
    updating different points concurrently.   
               
    \item[\textbf{C} -] Tasks will have to communicate with adjacent tasks after every
    update in order to get information needed in the stencil for the next
    update.
               
    \item[\textbf{A} -] In order to decrease the communication versus
    computation ratio, we can agglomerate the tasks of updating multiple points
    in the domain into one processor. We can minimize the ratio of comm v comp,
    using some basic calculus. This minimization will depend on just how long
    the ``extra physics'' of the problem is. For example, if the computations is
    rather expensive, this might affect the optimal agglomeration size. 
               
    \item[\textbf{M} -] If for example, the agglomeration size is too large or
    too small in order to partition the domain across the number of processors
    evenly, there are several ways to handle this. Some examples, would be for
    some processors to take a slightly larger domain (though this might break
    symmetry), or some processors may take an entire extra agglomerate compared
    to the other processors (bad if each agglomerate takes a considerable
    amount of time), or each processor can take a slightly suboptimal
    agglomeration in order to load balance the tasks evenly. 
               
\end{enumerate}
               
\section*{Question 2: PCAM for Node Search Problem}
\begin{enumerate}
               
    \item[\textbf{P} -] Smallest task is to search down to the bottom of a
    branch, i.e. a leaf and checking the area needed by that orientation.
    Different processors can investigate different leaves. 
               
    \item[\textbf{C} -] Tasks will have to communicate with other tasks in order
    to see which leaf minimizes the area. 
               
    \item[\textbf{A} -] If a particular branch doesn't have too many
    leaves/branches originating from it, then it might make sense for a single
    processor to search that branch. 
               
    \item[\textbf{M} -] In order to load balance on processors, it might make
    sense to probe 1 or two layers at each iteration to see how many leaves that
    node has. Overall, I think the best way to do this problem, is to have the
    root cpu to a maximum depth search, at each level picking the left most
    leaf. Then going back to the top, partitioning nodes to search into
    agglomerates for some arbitrary number of processors to compute. Then to go
    down 1 layer at a time, communicating at each layer, to see if anybody is
    above the current minimum in order to prune branches. Then tasks and
    agglomerates might be dynamically allocated to processors so that the load
    balancing would be updated and constant as the search progresses. If ever a
    new minimum is found at a certain layer search, then this should be
    broadcasted to the other processors. 

\end{enumerate}


\end{document}







