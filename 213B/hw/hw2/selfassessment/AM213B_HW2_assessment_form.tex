\documentclass[10pt,english]{article}
\usepackage{geometry}
 \geometry{verbose,tmargin=2cm,bmargin=2cm,footskip=1cm}
\usepackage{babel}
\usepackage{bm}
\usepackage{amssymb}
\usepackage{amsmath}
\usepackage{multirow}
\usepackage{graphicx}
\usepackage{setspace}
\onehalfspacing

\newcommand{\dt}{\Delta t}


\def\vss{\vspace{1cm}}
\def\vs{\vspace{0.2cm}}

\begin{document}

\noindent
\centerline{
\textbf{\large Numerical Methods for the Solution of Differential Equations (AM 213B)}}\\
\centerline{{\bf Homework 2 - grading form}}
%
\centerline{\line(1,0){480}}\vspace{.cm}

\vspace{0.2cm}
\flushleft{\bf Name:} Dante Buhl
\vs
\flushleft{\bf Final score:} 97/100

\vss\noindent
\centerline{\bf Point allocation explanation}


\vs
{\bf Question 1 (30/30 points):} Student found the correct value of $\dt^*$ up to 8
decimal places. The method described to obtain this value is conceptually
correct. An accurate plot for the region of absolute stability is included in
the PDF, and 3 different plots for the solutions of the ODE for varying values
of $\dt$. Plots are accurate and portray the conditional absolute stability
inherent to this method. 

\vs   
{\bf Question 2 (20/20 points):} Student references the Covergence Theorem and
uses Taylor expansion to show consistency with order three. Zero-stability is
also investigated correctly. Student correctly plots the Region of Absolute
Stability and arrives at the conclusion that BDF3 is not A-Stable for the same reason mentioned in
the solution. 

\vs
{\bf Question 3 (30/30 points):} Student correctly computes the consistency order
for the LMM and shows it is consistent. Zero-unstability is also shown correctly
and student arrives at the conclusion that the method is therefore not
convergent. Student also uses theoretical and numerical evidence to show that
this LMM is unconditionally absolutely unstable. 


\vs
{\bf Question 4 (17/20 points):} Student does not write out the RK3 scheme
explicitly. Student shows consistency, zero-stability, and convergence for the
RK3 method. The method of proving consistency given in the solution is contained
in the students work. Student confirms that the RK3 method is in fact not
A-stable, by showing that its region of absolute stability does not contain in
whole $\mathbb{C}^{-1}$. Plot given for the region of absolute stability is
messy, but its caption clarifies where the region of asbolute stablility is
given and it is an accurate representation. 






\end{document}
