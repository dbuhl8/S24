\documentclass{article}
\usepackage{graphicx} % Required for inserting images
\usepackage[margin=1in]{geometry}
\usepackage{amsmath}
\usepackage{amsthm}
\usepackage{amssymb}
\usepackage{amsfonts}
\usepackage{enumitem}
\usepackage{verbatim}
\usepackage{xcolor}

\title{Homework 2: Report}
\author{Dante Buhl}
\date{April. $29^{th}$ 2024}


\DeclareMathOperator{\cond}{cond}
\DeclareMathOperator{\vecspan}{span}

\begin{document}

\newcommand{\bs}[1]{\boldsymbol{#1}}
\newcommand{\bmp}[1]{\begin{minipage}{#1\textwidth}}
\newcommand{\emp}{\end{minipage}}
\newcommand{\R}{\mathbb{R}}
\newcommand{\C}{\mathbb{C}}
\newcommand{\N}{\mathcal{N}}
\newcommand{\I}{\mathrm{I}}
\newcommand{\K}{\bs{\mathrm{K}}}
\newcommand{\m}{\bs{\mu}_*}
\newcommand{\s}{\bs{\Sigma}_*}
\newcommand{\dt}{\Delta t}
\newcommand{\tr}[1]{\text{Tr}(#1)}
\newcommand{\Tr}[1]{\text{Tr}(#1)}
      
\maketitle

\section*{Problem 1: Absolute Stability for AB3}
\begin{enumerate}[label=\alph*)]

  \item Determine the largest value of $\Delta t$, for which the three-step 
        Adams-Bashforth method (AB3)
    \begin{proof}
      We use the condition for absolute stability: 
      \begin{align}
        \lim_{k \to \infty} ||\bs{u}_k|| = 0
      \end{align}
      For this specific numerical method, we have the following characteristic
      polynomial for the numerical method. (Note that since the columns of $B$
      are linearly independent we have that $B$ is diagonalizable).
      \begin{align}
        \bs{u}_{k+3} = \bs{u}_{k+2} + \frac{\dt}{12}\left(23\bs{f}_{k+2} -
        16\bs{f}_{k+1} + 5\bs{f}_k\right)\\
        \bs{u}_{k+3} - \bs{u}_{k+2} = \frac{\dt}{12}\left( 23\bs{Au}_{k+2} -
        16\bs{Au}_{k+1} + 5\bs{Au}_k\right)\\
        \bs{w}_{k+3} - \bs{w}_{k+2} = \frac{\dt}{12}\Lambda\left( 23\bs{w}_{k+2} -
        16\bs{w}_{k+1} + 5\bs{w}_k\right)
      \end{align}
      From this form of the Adams Bashforth method, we have that the
      coefficients $\alpha_i$ and $\beta_i$ are as follows, 
      \begin{align}
        \bs{\alpha} = [0, 0, -1, 1], \quad \bs{\beta} = \left[ \frac{5}{12},
        -\frac{16}{12}, \frac{23}{12}, 0\right]  \\
        \sum_{i=0}^3 (\alpha_i - \dt\lambda_m\beta_i)\bs{w}_{k+i}^m = 0
      \end{align}
      At this point, bother to find the eigenvalues of the matrix A which form
      $\Lambda$. Using a matlab eigenvalue solver, we find the eigenvalues of
      $A$ to be, 
      \begin{align}
            \lambda \approx \left[ -0.9667 \pm i 30.1255, -99.0667\right]\\
            \R(\lambda) \approx \left[ -0.9667, -99.0667\right]
      \end{align}
      We also only consider the real part of $\lambda$ as this is what will
      contribute to the convergence/stability. At this point we have 2 equations
      to solve in order to find the requirement on $\dt$ for the absolute
      convergence. The two equations are related to the characteristic
      polynomial for the iteration process. 
      \begin{align}
        \pi(z) = p(z) - \dt\lambda_i\sigma(z) = 0\\
        p(z) = \sum_{j=0}^{q} \alpha_j z^j, \quad 
        \sigma(z) = \sum_{j=0}^{q} \beta_j z^j
      \end{align}   
      We solve $(9)$ twice in order, once for each eigenvalue of our original
      linear transformation, $A$. The exact polynomial becomes,
      \begin{align}
        z^3 - z^2 -\frac{dt\lambda}{12}\left(\frac{5}{12} - \frac{16}{12}z +
        \frac{23}{12}z^2\right) &= 0\\
        z^3 - z^2\left(1 + \frac{23\dt\lambda}{12}\right) +
        \frac{16\dt\lambda}{12}z - \frac{5\dt\lambda}{12} &= 0
        \end{align}
        \begin{align}
        \begin{split}
         z^3 - z^2\left(1 + \frac{23\dt(-0.9667)}{12}\right) +
        \frac{16\dt(-0.9667)}{12}z - \frac{5\dt(-0.9667)}{12} &= 0\\
          z^3 - z^2\left(1 + \frac{23\dt(-99.0667)}{12}\right) +
        \frac{16\dt(-99.0667)}{12}z - \frac{5\dt(-99.0667)}{12} &= 0
        \end{split}
      \end{align}
    \end{proof}
  \item 

\end{enumerate}


\section*{Question 2: Convergence and Asbolute Stability for the BDF3 Method}

Criterion for Consistency for a lineaer multistep method
\begin{align}
\rho(1) &=0 \\
\rho(1) - \sigma(1) &=0
\end{align}

\begin{enumerate}[label=\alph*)]

  \item       

\end{enumerate}
\section*{Question 3: Consistency, Convergence, and Stability for an LMM}

\section*{Question 4: Convergence and Stability for an RK Method}


\end{document}
